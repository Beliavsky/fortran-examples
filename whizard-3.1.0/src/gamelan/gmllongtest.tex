% gmllongtest.tex --
% Test for gamelan.mp
\documentclass[10pt]{article}
\usepackage{gamelan}
\begin{document}
\begin{gmlfile}
\begin{gmlcode}
  message "These should be 1, 20, 300, 4e10, 5e-13, -6, -7e8, -8e-2.";
  showfloat #1, #20, #300, #4e10, #5e-13, #-6, #-7e8, #-8e-2;
  message "This should be 1.2345678.";
  showfloat #1.2345678;
\end{gmlcode}
\gmlcolor{snow}                     {"255 250 250"}
\gmlcolor{ghost white}              {"248 248 255"}
\gmlcolor{white smoke}              {"245 245 245"}
\gmlcolor{gainsboro}                {"220 220 220"}
\gmlcolor{floral white}             {"255 250 240"}
\gmlcolor{old lace}                 {"253 245 230"}
\gmlcolor{linen}                    {"250 240 230"}
\gmlcolor{antique white}            {"250 235 215"}
\gmlcolor{papaya whip}              {"255 239 213"}
\gmlcolor{blanched almond}          {"255 235 205"}
\gmlcolor{bisque}                   {"255 228 196"}
\gmlcolor{peach puff}               {"255 218 185"}
\gmlcolor{navajo white}             {"255 222 173"}
\gmlcolor{moccasin}                 {"255 228 181"}
\gmlcolor{cornsilk}                 {"255 248 220"}
\gmlcolor{ivory}                    {"255 255 240"}
\gmlcolor{lemon chiffon}            {"255 250 205"}
\gmlcolor{seashell}                 {"255 245 238"}
\gmlcolor{honeydew}                 {"240 255 240"}
\gmlcolor{mint cream}               {"245 255 250"}
\gmlcolor{azure}                    {"240 255 255"}
\gmlcolor{alice blue}               {"240 248 255"}
\gmlcolor{lavender}                 {"230 230 250"}
\gmlcolor{lavender blush}           {"255 240 245"}
\gmlcolor{misty rose}               {"255 228 225"}
\gmlcolor{white}                    {"255 255 255"}
\gmlcolor{black}                    {"000 000 000"}
\gmlcolor{dark slate gray}          {"047 079 079"}
\gmlcolor{dim gray}                 {"105 105 105"}
\gmlcolor{slate gray}               {"112 128 144"}
\gmlcolor{light slate gray}         {"119 136 153"}
\gmlcolor{medium gry}               {"190 190 190"}
\gmlcolor{light gray}               {"211 211 211"}
\gmlcolor{midnight blue}            {"025 025 112"}
\gmlcolor{navy}                     {"000 000 128"}
\gmlcolor{navy blue}                {"000 000 128"}
\gmlcolor{cornflower blue}          {"100 149 237"}
\gmlcolor{dark slate blue}          {"072 061 139"}
\gmlcolor{slate blue}               {"106 090 205"}
\gmlcolor{medium slate blue}        {"123 104 238"}
\gmlcolor{light slate blue}         {"132 112 255"}
\gmlcolor{medium blue}              {"000 000 205"}
\gmlcolor{royal blue}               {"065 105 225"}
\gmlcolor{blue}                     {"000 000 255"}
\gmlcolor{dodger blue}              {"030 144 255"}
\gmlcolor{deep sky blue}            {"000 191 255"}
\gmlcolor{sky blue}                 {"135 206 235"}
\gmlcolor{light sky blue}           {"135 206 250"}
\gmlcolor{steel blue}               {"070 130 180"}
\gmlcolor{light steel blue}         {"176 196 222"}
\gmlcolor{light blue}               {"173 216 230"}
\gmlcolor{powder blue}              {"176 224 230"}
\gmlcolor{pale turquoise}           {"175 238 238"}
\gmlcolor{dark turquoise}           {"000 206 209"}
\gmlcolor{medium turquoise}         {"072 209 204"}
\gmlcolor{turquoise}                {"064 224 208"}
\gmlcolor{cyan}                     {"000 255 255"}
\gmlcolor{light cyan}               {"224 255 255"}
\gmlcolor{cadet blue}               {"095 158 160"}
\gmlcolor{medium aquamarine}        {"102 205 170"}
\gmlcolor{aquamarine}               {"127 255 212"}
\gmlcolor{dark green}               {"000 100 000"}
\gmlcolor{dark olive green}         {"085 107 047"}
\gmlcolor{dark sea green}           {"143 188 143"}
\gmlcolor{sea green}                {"046 139 087"}
\gmlcolor{medium sea green}         {"060 179 113"}
\gmlcolor{light sea green}          {"032 178 170"}
\gmlcolor{pale green}               {"152 251 152"}
\gmlcolor{spring green}             {"000 255 127"}
\gmlcolor{lawn green}               {"124 252 000"}
\gmlcolor{green}                    {"000 255 000"}
\gmlcolor{chartreuse}               {"127 255 000"}
\gmlcolor{medium spring green}      {"000 250 154"}
\gmlcolor{green yellow}             {"173 255 047"}
\gmlcolor{lime green}               {"050 205 050"}
\gmlcolor{yellow green}             {"154 205 050"}
\gmlcolor{forest green}             {"034 139 034"}
\gmlcolor{olive drab}               {"107 142 035"}
\gmlcolor{dark khaki}               {"189 183 107"}
\gmlcolor{khaki}                    {"240 230 140"}
\gmlcolor{pale goldenrod}           {"238 232 170"}
\gmlcolor{light goldenrod yellow}   {"250 250 210"}
\gmlcolor{light yellow}             {"255 255 224"}
\gmlcolor{yellow}                   {"255 255 000"}
\gmlcolor{gold}                     {"255 215 000"}
\gmlcolor{light goldenrod}          {"238 221 130"}
\gmlcolor{goldenrod}                {"218 165 032"}
\gmlcolor{dark goldenrod}           {"184 134 011"}
\gmlcolor{rosy brown}               {"188 143 143"}
\gmlcolor{indian red}               {"205 092 092"}
\gmlcolor{saddle brown}             {"139 069 019"}
\gmlcolor{sienna}                   {"160 082 045"}
\gmlcolor{peru}                     {"205 133 063"}
\gmlcolor{burlywood}                {"222 184 135"}
\gmlcolor{beige}                    {"245 245 220"}
\gmlcolor{wheat}                    {"245 222 179"}
\gmlcolor{sandy brown}              {"244 164 096"}
\gmlcolor{medium tan}               {"210 180 140"}
\gmlcolor{chocolate}                {"210 105 030"}
\gmlcolor{firebrick}                {"178 034 034"}
\gmlcolor{brown}                    {"165 042 042"}
\gmlcolor{dark salmon}              {"233 150 122"}
\gmlcolor{salmon}                   {"250 128 114"}
\gmlcolor{light salmon}             {"255 160 122"}
\gmlcolor{orange}                   {"255 165 000"}
\gmlcolor{dark orange}              {"255 140 000"}
\gmlcolor{coral}                    {"255 127 080"}
\gmlcolor{light coral}              {"240 128 128"}
\gmlcolor{tomato}                   {"255 099 071"}
\gmlcolor{orange red}               {"255 069 000"}
\gmlcolor{red}                      {"255 000 000"}
\gmlcolor{hot pink}                 {"255 105 180"}
\gmlcolor{deep pink}                {"255 020 147"}
\gmlcolor{pink}                     {"255 192 203"}
\gmlcolor{light pink}               {"255 182 193"}
\gmlcolor{pale violet red}          {"219 112 147"}
\gmlcolor{maroon}                   {"176 048 096"}
\gmlcolor{medium violet red}        {"199 021 133"}
\gmlcolor{violet red}               {"208 032 144"}
\gmlcolor{magenta}                  {"255 000 255"}
\gmlcolor{violet}                   {"238 130 238"}
\gmlcolor{plum}                     {"221 160 221"}
\gmlcolor{orchid}                   {"218 112 214"}
\gmlcolor{medium orchid}            {"186 085 211"}
\gmlcolor{dark orchid}              {"153 050 204"}
\gmlcolor{dark violet}              {"148 000 211"}
\gmlcolor{blue violet}              {"138 043 226"}
\gmlcolor{purple}                   {"160 032 240"}
\gmlcolor{medium purple}            {"147 112 219"}
\gmlcolor{thistle}                  {"216 191 216"}
\gmlcolor{dark gray}                {"169 169 169"}
\gmlcolor{dark blue}                {"000 000 139"}
\gmlcolor{dark cyan}                {"000 139 139"}
\gmlcolor{dark magenta}             {"139 000 139"}
\gmlcolor{dark red}                 {"139 000 000"}
\gmlcolor{light green}              {"144 238 144"}

\begin{center}\unitlength1mm
\begin{gmlfigure}
tracingsets:=1; tracingonline:=1;
picture px; px =
begingraph(12cm,10cm)
  graphrange (#0,#1.7), (#10,??);
  fromfile "gmllongtest.dat": 

    % A curve to be plotted in one line
    table plot(u)(); showdata u;

    % data points with x and y error bars
    for l withinblock: get x,y,h; plot(t) z vbar h hbar (h over two); endfor
    showdata t;

    % Two named datasets
    tables plot(a1,a2)();
    showdata a1,a2;

    for l withinblock: get x,y; hist(s) (x, y plus #6); endfor
    showdata s;
  endfrom

  % A band between two curves. 
  fill from(a1|a2\) linked(smoothly,straight) 
    withbackground 
        (spectrum(50)(red,magenta,blue) xscaled width yscaled height)
    withlegend "Band";

  % A small histogram that has been read in as horizontal bars,
  % drawn with an offset in absolute coordinates.  The label points to
  % the midpoint of the 2nd bar, that is point#1 on part#1 of the path
  % set.  
  calculate bb(s) (x,#6);
  draw piecewise cyclic from(s,bb/\) shifted (5mm,0);
  phantom from(bb$1\) shifted (5mm,0)
    withlabel.bot("Histogram", on curve at 1);

  % A triangle in graph coordinates
  fill plot((#3,#4),(#9,#5),(#6,#4)) withcolor green
    withlegend "Triangle" outlined;

  % A curve with circles at the data points.  The first label refers to 
  % point#2 on the first drawn path, the second one to the point where
  % it intersects x=0.
  draw from(u) dashed evenly withcolor red
    withsymbol(circle scaled 3mm) withlegend "Curve"
    withdotlabel.urt("Maximum", on curve at 2)
    withdotlabel.lft(btex $x=0$ etex, on curve at (#0,??) shifted (-3mm,0));

  % A label at a point in graph coords
  dotlabel.lrt(image(drawarrow (1cm,-1cm)--origin;
                     label.lrt(btex $(8,6)$ etex, (1cm,-1cm))),
               on graph at (#8,#6));

  % Horizontal and vertical error bars, with square symbols
  picture sq; sq = 
    image(draw square scaled 3mm; fill square scaled 2mm);
  draw piecewise from(t) withcolor .5white withsymbol sq withticks 
    withlegend btex Error bars etex;

  % Another triangle, completely in absolute coords
  draw(3cm,4cm)--(2cm,5cm)--(1cm,1cm)--cycle 
    dashed withdots scaled .5 withpenscale 2
    withdotlabel.ulft("Corner", on curve at 2);

  % A histogram, calculated directly instead of using a file, drawn with
  % dots at the reference points.  The color has been read from rgb.txt
  hist(c) (#5,#4), (#5.5,#4.8), (#6,#5.4), (#6.5,#4.5), (#7,??);
  calculate cb(c) (x,#3.5);
  fill piecewise from(c,cb/\) hatched withstripes rotated 30 
    withcolor orange
    withshadow shifted (2mm,-3mm) withcolor .8[orange,white]
    withlegend "Histogram";
  phantom from(c) withsymbol(fshape(circle scaled 2mm)() colored red)
    withlegend "Values";

  % A closed shape
  draw cyclic plot(
    (#8,#2),(#9,#2.1),(#10,#2.5),(#9,#2.9),
    (#8,#3),(#7,#2.9),(#6,#2.5),(#7,#2.1))
    dashed withdashdots linked smoothly;

  % A label picture that shows up at the left margin (slightly offset)
  label.rt(image(drawarrow (1cm,0)--origin;
                 label.rt(btex$6.5$etex, (1cm,0))),
           on graph at (??,#6.5) shifted (3mm,0));

  % Data points with cumulative errors
  fromfile "gmllongtest.dat":
    for l withinblock: 
      get x,y,a,b; plot(r1) z vbar a; plot(r2) z vbar (a plus b);
    endfor
  endfrom;

  % Outer labels and tick marks
  label.top(btex\Large Test graph etex, out);
  grid(labeled noticks rt, 10 iticks top, small iticks top) withlength 6mm; 
  defaultgrid(2); grid.bot(small oticks);
  frame.llft;

  % The legend
  begingroup interim bboxmargin:=3mm;
    label.urt(fdbox(thelegend)
        (withcolor yellow withshadow shifted (2mm,-3mm) withcolor .8white)
            ()(), on graph at (#2,#5.2));
  endgroup;
endgraph;    
% Show the picture together with its bounding box
draw dbox(px)()();
\end{gmlfigure}
\end{center}
\end{gmlfile}
\end{document}
