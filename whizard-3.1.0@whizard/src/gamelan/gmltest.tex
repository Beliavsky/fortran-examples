% gmltest.tex --
% Test for gamelan.mp
\documentclass[12pt]{article}
\usepackage{gamelan}
\begin{gmlpreamble}
\usepackage{latexsym}
\def\foo{This is $\leadsto$}
\def\bar{some text}
\end{gmlpreamble}
\begin{document}
\begin{gmlfile}
\gml{tracingscale:=tracingsets:=1;}
\unitlength1mm
\begin{gmlgraph}(100,100)[a,b,c,d_u]
plot(a) (#0,#2), (#3,#1), (#6,#3), ??, (#7,#3), (#10,#9);
plot(b2) (#0,#1), (#2,#4), (#5,#5), ??, (#6,#7), (#9,#10);
picture s; s = fdshape(square scaled 2mm)(withcolor .9white)();
draw piecewise from(a) dashed evenly withsymbol s withlegend "A";
draw from(b2) withlegend "XXX" witharrows;
calculate c[-1][3.1](a,b2)(x, y1 plus y2);
freeze;
draw from(c[-1][3.1]) 
  withlabel("ABC", on curve at (#1,??)) 
  withlabel("DEF", on curve at 2.5)
  withsymbol (pentagram scaled 3mm);
plot(d_u.4vv r) (#8,#2), (#10,#2), (#10,#3);
fill from(d_u.4vv r) withcolor yellow outlined dashed withdashdots
  withlegend "QQQ";
fill from(d_u.4vv r) shifted (-5mm,-5mm) 
        hatched withstripes scaled 2 rotated -30 withcolor red
        outlined withlegend "qqq";
interim tracingonline:=1;
label.urt(thelegend,(1cm,7cm));
label.lft(<<\foo\bar>> rotated 90)(out);
line.bot("",#5);
\end{gmlgraph}
\gml{show currentpicture;}

\foo\bar.
\end{gmlfile}
\end{document}
